\nonstopmode
\documentstyle[11pt]{article}
\setlength{\textheight}{9.0in}
\setlength{\textwidth}{6.5in}
\hoffset=-1.3 truecm
\voffset=-1.7 truecm


\newcommand{\Dt}[0]{\bigtriangleup t}
\newcommand{\Dx}[0]{\bigtriangleup x}
\makeatletter                                            %KT
\@ifundefined{epsfbox}{\@input{epsf.sty}}{\relax}        %KT
\def\plotone#1{\centering \leavevmode                    %KT
\epsfxsize=\columnwidth \epsfbox{#1}}                    %KT
\def\plotone_reduction#1#2{\centering \leavevmode        %KT
\epsfxsize=#2\columnwidth \epsfbox{#1}}                  %KT
\def\plottwo#1#2{\centering \leavevmode                  %KT
\epsfxsize=.45\columnwidth \epsfbox{#1} \hfil            %KT
\epsfxsize=.45\columnwidth \epsfbox{#2}}                 %KT
\def\plotfiddle#1#2#3#4#5#6#7{\centering \leavevmode     %KT
\vbox to#2{\rule{0pt}{#2}}                               %KT
\special{psfile=#1 voffset=#7 hoffset=#6 vscale=#5 hscale=#4 angle=#3}} %KT
\makeatother


\begin{document}
\section{Introduction}

Need for numerical methods.  ZEUS and other codes.  Reasons for a new code.
Why use conservative form.

The equations of motion can be written in conservative form as
\begin{equation}
\frac{\partial {\bf U}}{\partial t} + 
   {\bf \nabla} \cdot {\bf F}({\bf U}) = 0,
\end{equation}
where ${\bf U}$ is a vector of conserved quantities and ${\bf F}$ the vector
of their fluxes.  The number and form of the components of both ${\bf U}$
and ${\bf F}$ depend on the physics of the problem to be solved.  

Outline of paper.

\section{The Numerical Algorithm in One-Dimension}

\subsection{Basic Equations}

For adiabatic MHD (the most general system considered here),
the vector of conserved quantities and their fluxes are
\begin{equation}
{\bf U}  = \left[ \begin{array}{c}
  \rho \\
  \rho v_{x} \\
  \rho v_{y} \\
  \rho v_{z} \\
   E \\
   B_{y} \\
   B_{z} \\ \end{array} \right] , \hspace{1cm}
{\bf F_{x}} = \left[ \begin{array}{c}
  \rho v_{x} \\
  \rho v_{x}^{2} + P + B^{2}/(8\pi) - B_{x}^{2}/(4\pi) \\
  \rho v_{x}v_{y} - B_{x}B_{y}/(4\pi) \\
  \rho v_{x}v_{z} - B_{x}B_{z}/(4\pi) \\
  (E + P + B^{2}/(8\pi)) - ({\bf B}\cdot{\bf v})B_{x}/(4\pi) \\
  B_{y}v_{x} - B_{x}v_{y} \\
  B_{z}v_{x} - B_{x}v_{z} \end{array} \right].
\end{equation}
where
\begin{equation}
  E = \frac{P}{\gamma -1} + \frac{1}{2}\rho v^{2} + \frac{B^{2}}{8\pi}
\end{equation}
and $B^{2} = {\bf B} \cdot {\bf B}$.

For isothermal MHD, the fifth components of ${\bf U}$ and ${\bf F_{x}}$
are dropped, and equation (3) is replaced with $P=\rho C^{2}$, where
$C$ is a constant (the isothermal speed of sound).  For adiabatic
hydrodynamics, the last two components of ${\bf U}$ and ${\bf F_{x}}$
are dropped, as well as all terms involving the magnetic field in
equations (2) and (3).  For isothermal hydrodynamics, the last three
components of ${\bf U}$ and ${\bf F_{x}}$ are dropped, as well as all
terms involving the magnetic field in equations (2) and (3).

\subsection{Discretization}

To develop a numerical algorithm which generates approximate solutions to
equation (1), we first discretize each variable on a spatial mesh.
These are zone averages.

\begin{figure}
\plotone_reduction{fig1.eps}{0.8}
\caption{Temporal and spatial centering of variables in 1-D}
\end{figure}

A temporally second-order accurate representation of equation (1) is then
\begin{equation}
  {\bf U}_{i}^{n+1} = {\bf U}_{i}{n} - 
\frac{\Dt}{\Dx}\left( {\bf F}_{i+1/2}^{n+1/2} - {\bf F}_{i-1/2}^{n+1/2} \right)
\end{equation}

Most of the effort in the numerical scheme goes into developing an
accurate and stable form for the time-centered fluxes 
${\bf F}_{i\pm 1/2}^{n+1/2}$.

Discussion of the basic types of schemes used to compute fluxes.
Centered difference schemes. Godunov schemes.  Riemann solvers.
Linearized solvers. $U_{,t} = AU_{,x}$.
Roe (1981) has developed one linearization.  Why we choose Roe's method.

However, we do not really need an accurate solution for the fluxes as a
function of time.  What we are most interested in is an accurate representation
of the integrated flux through each cell interface over one timestep.
Thus, the most attractive schemes are those which use a simple construction
for the fluxes which is nonetheless accurate for the integral.

\subsection{Roe's Method}

Property U.  Parameter vector.

The fluxes are
\begin{equation}
{\bf F}_{i+1/2}^{n+1/2} = \frac{1}{2} \left( {\bf F}_{L} + {\bf F}_{R} \right)
 + \frac{1}{2} \Sigma a_{i} | \lambda_{i} | {\bf R}_{i}
\end{equation}
where ${\bf F}_{L}={\bf F}({\bf U}_{L})$, ${\bf F}_{R}={\bf F}({\bf U}_{R})$,
and
\begin{equation}
  a_{i} = {\bf L}_{i} \bigtriangleup {\bf U}_{i}
\end{equation}
\begin{equation}
  \bigtriangleup {\bf U}_{i} = {\bf U}_{i,L} - {\bf U}_{i,R}
\end{equation}
and $\lambda_{i}$, ${\bf R}_{i}$, and ${\bf L}_{i}$ are the eigenvalues,
and right- and left-eigenvectors of the linearized system respectively.
These are functions of the primitive variables, that is
$\lambda_{i} = \lambda_{i}(\bar{\bf U})$,
${\bf R}_{i} = {\bf R}_{i}(\bar{\bf U})$, and
${\bf L}_{i} = {\bf L}_{i}(\bar{\bf U})$.

The functional forms for the eigenvalues and eigenvectors of course
depend on the physics.  Explicit forms for each are given in an Appendix
for the cases of adiabatic or isothermal hydrodynamics or MHD.
Many choices are possible for, the average state $\bar{\bf U}$.  Roe (1981)
has shown that one optimal choice is given by...

\begin{figure}
\plotone_reduction{fig2.ps}{0.8}
\caption{Representation of Riemann problem with Roe's linearization.}
\end{figure}

\begin{figure}
\plotone_reduction{fig3.ps}{0.8}
\caption{Cokmparison of exact and approximate solution to Sod's shocktube
problem.}
\end{figure}

The ${\bf U}_{L}$ and ${\bf U}_{R}$ are the left- and right-state
constructed by a spatial interpolation of the distribution of U within
each cell; the technique for claculating these is described below.

\subsection{Left and Right States}

Spatial interpolation of primitive versus characteristic variables
(latter is supposedly superior).

Following Colella (1990), the left- and right-state vectors are
\begin{equation}
{\bf U}_{L,i+1/2}^{n+1/2} = \bar{\bf U}_{L,i+1/2} + \frac{\Dt}{2 \Dx}
\Sigma (\lambda_{i}^{M} - \lambda_{i}^{\alpha})
 \bigtriangleup W_{i} R({\bf U}_{i})
\end{equation}
\begin{equation}
{\bf U}_{R,i+1/2}^{n+1/2} = \bar{\bf U}_{R,i+1/2} + \frac{\Dt}{2 \Dx}
\Sigma (\lambda_{i}^{m} - \lambda_{i}^{\alpha})
 \bigtriangleup W_{i+1} R({\bf U}_{i+1})
\end{equation}
where
\begin{equation}
\bar{\bf U}_{i+1/2,L} = {\bf U}_{i} + \left[ \frac{1}{2} - 
 max (\lambda_{i},0)\frac{\Dt}{2 \Dx} \right] \bigtriangleup {\bf U}_{i}
\end{equation}
\begin{equation}
\bar{\bf U}_{i+1/2,R} = {\bf U}_{i+1} - \left[ \frac{1}{2} +
 min (\lambda_{i+1},0)\frac{\Dt}{2 \Dx} \right] \bigtriangleup {\bf U}_{i+1}
\end{equation}
and
\begin{equation}
\bigtriangleup {\bf U}_{i} = \Sigma \bigtriangleup W_{i} R({\bf U}_{i})
\end{equation}

\subsection{Time step Criterion}

\subsection{Tests}

Hydro: advection (+ convergence rates); Sod (+ errors compared to ZEUS, Lagrange plus remap PPM); Einfeldt, two interacting blast waves.

MHD: Brio \& Wu, Alfven waves, all four R\&J problems.

\section{The Numerical Algorithm in Multidimensions}

\subsection{Tests}

\section{Nested Grids}

\subsection{Tests}

\appendix
\setcounter{equation}{0}
\renewcommand{\theequation}{\Alph{section}\arabic{equation}}
\section{Eigensystems in the Primitive Variables}

In this appendix, we give explicit forms for the the eigenvalues and
eigenvectors of the matrix ${\sf A}$ resulting from linearizing the
dynamical equations as ${\bf W}_{,t} = {\sf A}{\bf W}_{,x}$, where
${\bf W}$ is a vector composed of the primitive variables.  These
eigensystems are needed for the characteristic interpolation of the
primitive variables (see \S X).

\subsection{Adiabatic Hydrodynamics}

For adiabatic hydrodynamics, ${\bf W} = (\rho, v_{x}, v_{y}, v_{z}, P)$,
and the matrix ${\sf A}$ is
\begin{equation}
{\sf A} = \left[ \begin{array}{ccccc}
v_{x} & \rho       & 0     & 0     & 0      \\
0     & v_{x}      & 0     & 0     & 1/\rho \\
0     & 0          & v_{x} & 0     & 0      \\
0     & 0          & 0     & v_{x} & 0      \\
0     & \rho a^{2} & 0     & 0     & v_{x}  \end{array} \right] ,
\end{equation}
where $a^{2} = \gamma P/\rho$ ($a$ is the adiabatic sound speed).
The five eigenvalues of this matrix in ascending order are
\begin{equation}
{\bf \lambda} = (v_{x}-a, v_{x}, v_{x}, v_{x}, v_{x}+a).
\end{equation}
The corresponding right-eigenvectors are the columns of the matrix
\begin{equation}
{\sf R}  = \left[ \begin{array}{ccccc}
1       & 1 & 0 & 0 & 1      \\
-a/\rho & 0 & 0 & 0 & a/\rho \\
0       & 0 & 1 & 0 & 0      \\
0       & 0 & 0 & 1 & 0      \\
a^{2}   & 0 & 0 & 0 & a^{2}  \end{array} \right] ,
\end{equation}
while the left-eigenvectors are the rows of the matrix
\begin{equation}
{\sf L}  = \left[ \begin{array}{ccccc}
0 & -\rho/(2a) & 0 & 0 & 1/(2a^{2}) \\
1 &  0       & 0 & 0 & -1/a^{2} \\
0 &  0       & 1 & 0 &  0       \\
0 &  0       & 0 & 1 &  0       \\
0 & \rho/(2a)  & 0 & 0 & 1/(2a^{2}) \end{array} \right] .
\end{equation}

\subsection{Isothermal Hydrodynamics}

For isothermal hydrodynamics, ${\bf W} = (\rho, v_{x}, v_{y}, v_{z})$, and the
matrix ${\sf A}$ is identical to equation (A1), with the fourth row and column
dropped, and $a$ replaced by the isothermal sound speed $C$.
The four eigenvalues of the resulting matrix in ascending order are
\begin{equation}
{\bf \lambda} = (v_{x}-C, v_{x}, v_{x}, v_{x}+C).
\end{equation}
The corresponding right-eigenvectors are the columns of the matrix given
in equation (A3), while the left-eigenvectors are the rows of the matrix given
in equation (A4), where in both cases the fourth row and column is dropped,
and $a$ is replaced by $C$.

\subsection{Adiabatic Magnetohydrodynamics}

For adiabatic MHD, ${\bf W} = (\rho, v_{x}, v_{y}, v_{z}, P, b_{y}, b_{z})$,
where ${\bf b} = {\bf B}/\sqrt{4\pi}$, and the matrix ${\sf A}$ is
\begin{equation}
{\sf A} = \left[ \begin{array}{cccccccc}
v_x & \rho & 0 & 0 & 0 & 0 & 0 \\
0 & v_x & 0 & 0 & 1/\rho & b_y/\rho & b_z/\rho \\
0 & 0 & v_x & 0 & 0 & -b_x/\rho & 0 \\
0 & 0 & 0 & v_x & 0 & 0 & -b_x/\rho \\
0 & \rho a^{2} & 0 & 0 & v_{x} & 0 & 0 \\
0 & b_y & -b_x & 0 & 0 & v_x & 0 \\
0 & b_z & 0 & -b_x & 0 & 0 & v_x  \end{array} \right] .
\end{equation}
where $a^{2} = \gamma P/\rho$.
The seven eigenvalues of this matrix in ascending order are
\begin{equation}
{\bf \lambda} = (v_{x}-C_{f}, v_{x}-C_{Ax}, v_{x}-C_{s}, v_{x},
v_{x}+C_{s}, v_{x}+C_{Ax}, v_{x}+C_{f})
\end{equation}
where the fast- and slow-magnetosonic wave speeds are given by
\begin{equation}
C_{f,s}^2 = \frac{1}{2} \left( \left[a^{2}+C_{A}^{2}\right] \pm
\sqrt{\left[a^{2}+C_{A}^{2}\right]^{2} - 4 a^2 C_{Ax}^2}\right)
\end{equation}
(with $C_{f}[C_{s}]$ given by the $+[-]$ sign).  The Alfv\'{e}n speeds are
given by
\begin{equation}
C_{A}^{2} = (b_{x}^{2}+b_{y}^{2 }+b_{z}^{2})/\rho, \hspace*{1cm}
C_{Ax}^{2} = b_{x}^{2}/\rho.
\end{equation}
The corresponding right-eigenvectors are the columns of the matrix
\begin{equation}
{\sf R}  = \left[ \begin{array}{ccccccc}
\rho\alpha_{f} & 0 & \rho\alpha_{s} & 1 & \rho\alpha_{s} & 0 & \rho\alpha_{f} \\
-C_{ff} & 0 & -C_{ss} & 0 & C_{ss} & 0 & C_{ff} \\
Q_{s}\beta_{y} & -\beta_{z} & -Q_{f}\beta_{y} & 0 & Q_{f}\beta_{y} & \beta_{z} & -Q_{s}\beta_{y} \\
Q_{s}\beta_{z} & \beta_{y} & -Q_{f}\beta_{z} & 0 & Q_{f}\beta_{z} & -\beta_{y} & -Q_{s}\beta_{z} \\
\rho a^{2}\alpha_{f} & 0 & \rho a^{2}\alpha_{s} & 0 & \rho a^{2}\alpha_{s} & 0 & \rho a^{2}\alpha_{f} \\
A_{s}\beta_{y} & -\beta_{z}S\sqrt{\rho} & -A_{f}\beta_{y} & 0 & -A_{f}\beta_{y} & -\beta_{z}S\sqrt{\rho} & A_{s}\beta_{y} \\
A_{s}\beta_{z} & \beta_{y}S\sqrt{\rho} & -A_{f}\beta_{z} & 0 & -A_{f}\beta_{z} & \beta_{y}S\sqrt{\rho} & A_{s}\beta_{z} \end{array} \right] ,
\end{equation}
where $S={\rm sign}(b_{x})$, and
\begin{equation}
C_{ff} = C_{f}\alpha_{f}, \hspace*{1cm}
C_{ss} = C_{s}\alpha_{s},
\end{equation}
\begin{equation}
Q_{f} = C_{f}\alpha_{f}S, \hspace*{1cm}
Q_{s} = C_{s}\alpha_{s}S,
\end{equation}
\begin{equation}
A_{f} = a\alpha_{f}\sqrt{\rho}, \hspace*{1cm}
A_{s} = a\alpha_{s}\sqrt{\rho},
\end{equation}
\begin{equation}
\alpha_{f}^{2} = \frac{a^{2} - C_{s}^{2}}{C_{f}^{2} - C_{s}^{2}}, \hspace*{1cm}
\alpha_{s}^{2} = \frac{C_{f}^{2} - a^{2}}{C_{f}^{2} - C_{s}^{2}},
\end{equation}
\begin{equation}
\beta_{y} = \frac{b_{y}}{\sqrt{b_{y}^{2} + b_{z}^{2}}}, \hspace*{1cm}
\beta_{z} = \frac{b_{z}}{\sqrt{b_{y}^{2} + b_{z}^{2}}}.
\end{equation}
In the degenerate case where $C_{A}=C_{Ax}=a$, so that $C_{f}=C_{s}$, then
equation (A14) becomes $\alpha_{f}=1$ and $\alpha_{s}=0$.
The left-eigenvectors are the rows of the matrix
\begin{equation}
{\sf L}  = \left[ \begin{array}{ccccccc}
0 & -N_{f}C_{ff} & N_{f}Q_{s}\beta_{y} & N_{f}Q_{s}\beta_{z} & N_{f}\alpha_{f}/\rho & N_{f}A_{s}\beta_{y}/\rho & N_{f}A_{s}\beta_{z}/\rho \\
0 & 0 & -\beta_{z}/2 & \beta_{y}/2 & 0 & -\beta_{z}S/(2\sqrt{\rho}) & \beta_{y}S/(2\sqrt{\rho}) \\
0 & -N_{s}C_{ss} & -N_{s}Q_{f}\beta_{y} & -N_{s}Q_{f}\beta_{z} & N_{s}\alpha_{s}/\rho & -N_{s}A_{f}\beta_{y}/\rho & -N_{s}A_{f}\beta_{z}/\rho \\
1 & 0 & 0 & 0 & -1/a^{2} & 0 & 0 \\
0 & N_{s}C_{ss} & N_{s}Q_{f}\beta_{y} & N_{s}Q_{f}\beta_{z} & N_{s}\alpha_{s}/\rho & -N_{s}A_{f}\beta_{y}/\rho & -N_{s}A_{f}\beta_{z}/\rho \\
0 & 0 & \beta_{z}/2 & -\beta_{y}/2 & 0 & -\beta_{z}S/(2\sqrt{\rho}) & \beta_{y}S/(2\sqrt{\rho}) \\
0 & N_{f}C_{ff} & -N_{f}Q_{s}\beta_{y} & -N_{f}Q_{s}\beta_{z} & N_{f}\alpha_{f}/\rho & N_{f}A_{s}\beta_{y}/\rho & N_{f}A_{s}\beta_{z}/\rho \end{array} \right] ,
\end{equation}
where
\begin{equation}
N_{f} = N_{s} = \frac{1}{2a^{2}}
\end{equation}
are normalization factors for the eigenvectors corresponding to the fast-
and slow-magnetosonic waves respectively.

\subsection{Isothermal Magnetohydrodynamics}

For isothermal MHD, ${\bf W} = (\rho, v_{x}, v_{y}, v_{z}, b_{y}, b_{z})$,
where ${\bf b} = {\bf B}/\sqrt{4\pi}$, and the matrix ${\sf A}$ is
\begin{equation}
{\sf A} = \left[ \begin{array}{ccccccc}
v_x & \rho & 0 & 0 & 0 & 0 \\
0 & v_x & 0 & 1/\rho & b_y/\rho & b_z/\rho \\
0 & 0 & v_x & 0 & -b_x/\rho & 0 \\
0 & 0 & 0 & v_x & 0 & -b_x/\rho \\
0 & b_y & -b_x & 0 & v_x & 0 \\
0 & b_z & 0 & -b_x & 0 & v_x  \end{array} \right] .
\end{equation}
The six eigenvalues of this matrix in ascending order are
\begin{equation}
{\bf \lambda} = (v_{x}-C_{f}, v_{x}-C_{Ax}, v_{x}-C_{s}, 
v_{x}+C_{s}, v_{x}+C_{Ax}, v_{x}+C_{f}), 
\end{equation}
where the fast and slow-magnetosonic wave speeds are given by equation
(A8) (with $a$ replaced by the isothermal sound speed $C$ here and throughout),
and the Alfven speeds are given by equation (A9).  The
corresponding right-eigenvectors are the columns of the matrix given in
equation (A10), with the fifth row and fourth column dropped.  The
left-eigenvectors are the rows of the matrix given in equation (A14),
with the fifth row and fourth column dropped, and with
\begin{equation}
N_{f} = \frac{1}{C^{2}(1+\alpha_{s}^{2})}, \hspace*{1cm}
N_{s} = \frac{1}{C^{2}(1+\alpha_{f}^{2})}.
\end{equation}

\setcounter{equation}{0}
\section{Eigensystems in the Conserved Variables}

In this appendix, we give explicit forms for the the eigenvalues and
eigenvectors of the matrix ${\sf A}$ resulting from linearizing the
dynamical equations as ${\bf U}_{,t} = {\sf A}{\bf U}_{,x}$, where
${\bf U}$ is a vector composed of the conserved variables.  These
eigensystems are needed to construct the fluxes of the conserved
quantities (see \S X).

\subsection{Adiabatic Hydrodynamics}

For adiabatic hydrodynamics, ${\bf U} = (\rho, \rho v_{x}, \rho v_{y},
\rho v_{z}, E)$, and the matrix ${\sf A}$ is
\begin{equation}
{\sf A}  = \left[ \begin{array}{ccccc}
0 & 1 & 0 & 0 & 0 \\
-v_{x}^{2} + \gamma^{\prime}v^{2}/2 & -(\gamma -3)v_{x} & -\gamma^{\prime}v_{y} & -\gamma^{\prime}v_{z} & \gamma^{\prime} \\
-v_{x}v_{y} & v_{y} & v_{x} & 0 & 0 \\
-v_{x}v_{z} & v_{z} & 0 & v_{x} & 0 \\
-v_{x}H + \gamma^{\prime}v_{x}v^{2}/2 & -\gamma^{\prime}v_{x}^{2} + H & -\gamma^{\prime}v_{x}v_{y} & -\gamma^{\prime}v_{x}v_{z} & \gamma v_{x} \end{array} \right]
\end{equation}
where the enthalpy $H = (E+P)/\rho$, $v^{2} = {\bf v}\cdot{\bf v}$, and
$\gamma^{\prime} = (\gamma -1)$.  The five eigenvalues of this matrix
in ascending order are
\begin{equation}
{\bf \lambda} = (v_{x}-a, v_{x}, v_{x}, v_{x}, v_{x}+a),
\end{equation}
where $a^{2} = (\gamma-1)(H-v^{2}/2) = \gamma P/\rho$ ($a$ is the
adiabatic sound speed).  The corresponding right-eigenvectors are the
columns of the matrix
\begin{equation}
{\sf R}  = \left[ \begin{array}{ccccc}
1         & 0     & 0     & 1       & 1         \\ 
v_{x} - a & 0     & 0     & v_{x}   & v_{x} + a \\
v_{y}     & 1     & 0     & v_{y}   & v_{y}     \\
v_{z}     & 0     & 1     & v_{z}   & v_{z}     \\
H-v_{x}a  & v_{y} & v_{z} & v^{2}/2 & H+v_{x}a  \end{array} \right] ,
\end{equation}
The left-eigenvectors are the rows of the matrix
\begin{equation}
{\sf L} =  \left[ \begin{array}{ccccc}
N_{a}(\gamma^{\prime}v^{2}/2 + v_{x}a) & -N_{a}(\gamma^{\prime}v_{x} + a) & -N_{a}\gamma^{\prime}v_{y} & -N_{a}\gamma^{\prime}v_{z} & N_{a}\gamma^{\prime} \\
-v_{y} & 0 & 1 & 0 & 0 \\
-v_{z} & 0 & 0 & 1 & 0 \\
1-N_{a}\gamma^{\prime}v^{2} & \gamma^{\prime}v_{x}/a^{2} & \gamma^{\prime}v_{y}/a^{2} & \gamma^{\prime}v_{z}/a^{2} & -\gamma^{\prime}/a^{2} \\
N_{a}(\gamma^{\prime}v^{2}/2 - v_{x}a) & -N_{a}(\gamma^{\prime}v_{x} - a) & -N_{a}\gamma^{\prime}v_{y} & -N_{a}\gamma^{\prime}v_{z} & N_{a}\gamma^{\prime}
\end{array} \right] ,
\end{equation}
where $N_{a} = 1/(2a^{2})$.  These are identical to those given by Roe (1981),
except the second and third eigenvectors (corresponding to the transport of
shear motion) have been rescaled to avoid singularities.

\subsection{Isothermal Hydrodynamics}

For isothermal hydrodynamics, ${\bf U} = (\rho, \rho v_{x}, \rho v_{y},
\rho v_{z})$, and the matrix ${\sf A}$ is
\begin{equation}
{\sf A}  = \left[ \begin{array}{cccc}
0 & 1 & 0 & 0 \\
-v_{x}^{2} + C^{2} & 2v_{x} & 0 & 0 \\
-v_{x}v_{y} & v_{y} & v_{x} & 0 \\
-v_{x}v_{z} & v_{z} & 0 & v_{x} \end{array} \right]
\end{equation}
where $C$ is the isothermal sound speed.
The four eigenvalues of this matrix in ascending order are
\begin{equation}
{\bf \lambda} = (v_{x}-C, v_{x}, v_{x}, v_{x}+C).
\end{equation}
The corresponding right-eigenvectors are the columns of the matrix given
in equation (B3) with the fifth row and fourth column dropped, and $a$ replaced
by $C$ throughout.  The left-eigenvectors are the rows of the matrix
\begin{equation}
{\sf L}  = \left[ \begin{array}{cccc}
(1 + v_{x}/C)/2 & -1/(2C) & 0 & 0 \\
-v_{y}          &  0    & 1 & 0 \\
-v_{z}          &  0    & 0 & 1 \\
(1 - v_{x}/C)/2 & 1/(2C)  & 0 & 0 \end{array} \right] .
\end{equation}

\subsection{Adiabatic Magnetohydrodynamics}

For adiabatic MHD, ${\bf U} = (\rho, \rho v_{x}, \rho v_{y},
\rho v_{z}, E, b_{y}, b_{z})$, where ${\bf b} = {\bf B}/\sqrt{4\pi}$,
and the matrix ${\sf A}$ is
\begin{equation}
{\sf A} = \left[ \begin{array}{ccccccc}
0 & 1 & 0 & 0 & 0 & 0 & 0 \\
-v_{x}^{2} + \gamma^{\prime}v^{2}/2 - X^{\prime} & -(\gamma-3)v_{x} & -\gamma^{\prime}v_{y} & -\gamma^{\prime}v_{z} & \gamma^{\prime} & -b_{y}Y^{\prime} & -b_{z}Y^{\prime} \\
- v_{x} v_{y} & v_{y} & v_{x} & 0 & 0 & -b_{x} & 0 \\
- v_{x} v_{z} & v_{z} & 0 & v_{x} & 0 & 0 & -b_{x} \\
A_{51} & A_{52} & A_{53} & A_{54} & \gamma v_{x} & A_{56} & A_{57} \\
(b_{x}v_{y} - b_{y}v_{x})/\rho & b_{y}/\rho & -b_{x}/\rho & 0 & 0 & v_{x} & 0 \\
(b_{x}v_{z} - b_{z}v_{x})/\rho & b_{z}/\rho & 0 & -b_{x}/\rho & 0 & 0 & v_{x}
\end{array} \right]
\end{equation}
where $v^{2} = {\bf v}\cdot{\bf v}$, and
\begin{equation}
A_{51} = -v_{x}H + \gamma^{\prime}v_{x}v^{2}/2 + b_{x}(b_{x}v_{x} + b_{y}v_{y}
+ b_{z}v_{z})/\rho - v_{x}X^{\prime}
\end{equation}
\begin{equation}
A_{52} = -\gamma^{\prime}v_{x}^{2} + H - b_{x}^{2}/\rho
\end{equation}
\begin{equation}
A_{53} = -\gamma^{\prime}v_{x}v_{y} - b_{x}b_{y}/\rho
\end{equation}
\begin{equation}
A_{54} = -\gamma^{\prime}v_{x}v_{z} - b_{x}b_{z}/\rho
\end{equation}
\begin{equation}
A_{56} = -(b_{x}v_{y} + b_{y}v_{x}Y^{\prime})
\end{equation}
\begin{equation}
A_{57} = -(b_{x}v_{z} + b_{z}v_{x}Y^{\prime}) 
\end{equation}
\begin{equation}
X=\left[(b_{y}^2-b_{y,L}b_{y,R})+(b_{z}^2-b_{z,L}b_{z,R})\right]/(2\rho)
\end{equation}
\begin{equation}
Y = \frac{\rho_{L} + \rho_{R}}{2\rho}.
\end{equation}
In these equations
$\gamma^{\prime} = (\gamma - 1)$, $X^{\prime} = (\gamma-2)X$,
$Y^{\prime} = (\gamma-2)Y$, and $H=(E+P+b^{2}/2)/\rho$.
The factors $X$ and $Y$ are introduced by the averaging scheme defined by
equation (XX); thus the matrix ${\sf A}$ and its eigenvectors depend
explicitely on our choice of the Roe averaging scheme.
The seven eigenvalues of this matrix in ascending order are
\begin{equation}
{\bf \lambda} = (v_{x}-C_{f}, v_{x}-C_{Ax}, v_{x}-C_{s}, v_{x}
v_{x}+C_{s}, v_{x}+C_{Ax}, v_{x}+C_{f})
\end{equation}
where the fast and slow-magnetosonic wave speeds are given by
\begin{equation}
C_{f,s}^2 = \frac{1}{2} \left( \left[\tilde{a}^{2}+\tilde{C}_{A}^{2}\right] \pm
\sqrt{\left[\tilde{a}^{2}+\tilde{C}_{A}^{2}\right]^{2} - 
4\tilde{a}^2 C_{Ax}^2}\right)
\end{equation}
(with $C_{f}[C_{s}]$ given by the $+[-]$ sign), and
\begin{equation}
\tilde{a}^{2} = \gamma^{\prime}\left( H-v^{2}/2-b^{2}/\rho \right) - X^{\prime}
\end{equation}
\begin{equation}
\tilde{C}_{A}^{2} = C_{Ax}^{2} + b_{\perp}^{\ast 2}/\rho \hspace*{1cm}
C_{Ax}^{2} = b_{x}^{2}/\rho \hspace*{1cm}
b_{\perp}^{\ast 2} = (\gamma^{\prime} - Y^{\prime})(b_{y}^{2} + b_{z}^{2}).
\end{equation}
The corresponding right-eigenvectors are the columns of the matrix
\begin{equation}
{\sf R} = \left[ \begin{array}{ccccccc}
\alpha_{f} & 0 & \alpha_{s} & 1 &\alpha_{s} & 0 & \alpha_{f} \\
V_{xf}-C_{ff} & 0 & V_{xs}-C_{ss} & v_{x} & V_{xs}+C_{ss} & 0 & V_{xf}+C_{ff} \\
V_{yf}+Q_{s}\beta_{y}^{\ast} & -\beta_{z} & V_{ys}-Q_{f}\beta_{y}^{\ast} & v_{y} & V_{ys}+Q_{f}\beta_{y}^{\ast} & \beta_{z} & V_{yf}-Q_{s}\beta_{y}^{\ast} \\
V_{zf}+Q_{s}\beta_{z}^{\ast} & \beta_{y} & V_{zs}-Q_{f}\beta_{z}^{\ast} & v_{z} & V_{zs}+Q_{f}\beta_{z}^{\ast} & -\beta_{y} & V_{zf}-Q_{s}\beta_{z}^{\ast} \\
R_{51} & R_{52} & R_{53} & R_{54} & R_{55} & R_{56} & R_{57} \\ 
A_{s}\beta_{y}^{\ast}/\rho & -\beta_{z}S/\sqrt{\rho} & -A_{f}\beta_{y}^{\ast}/\rho & 0 & -A_{f}\beta_{y}^{\ast}/\rho & -\beta_{z}S/\sqrt{\rho} & A_{s}\beta_{y}^{\ast}/\rho \\
A_{s}\beta_{z}^{\ast}/\rho & \beta_{y}S/\sqrt{\rho} & -A_{f}\beta_{z}^{\ast}/\rho & 0 & -A_{f}\beta_{z}^{\ast}/\rho & \beta_{y}S/\sqrt{\rho} & A_{s}\beta_{z}^{\ast}/\rho \end{array} \right]
\end{equation}
where the $C_{ff,ss}, Q_{f,s}, A_{f,s}, \alpha_{f,s}$ and $\beta_{y,z}$ are
given by equation (A11) to (A13) (with $a$ replaced by $\tilde{a}$), 
$V_{if,s} = v_{i}\alpha_{f,s}$ $(i=x,y,z)$, and
\begin{equation}
R_{51} = \alpha_{f} \left( H^{\prime}- v_{x}C_{f} \right)
+Q_{s}(v_{y}\beta_{y}^{\ast} + v_{z}\beta_{z}^{\ast})
+A_{s}b_{\perp}^{\ast}\beta_{\perp}^{\ast 2}/\rho,
\end{equation}
\begin{equation}
R_{52} = -(v_{y}\beta_{z} - v_{z}\beta_{y}) = -R_{56},
\end{equation}
\begin{equation}
R_{53} = \alpha_{s} \left( H^{\prime} - v_{x}C_{s} \right)
-Q_{f}(v_{y}\beta_{y}^{\ast} + v_{z}\beta_{z}^{\ast})
-A_{f}b_{\perp}^{\ast}\beta_{\perp}^{\ast 2}/\rho,
\end{equation}
\begin{equation}
R_{54} = v^{2}/2 + X^{\prime}/\gamma^{\prime}
\end{equation}
\begin{equation}
R_{55} = \alpha_{s} \left( H^{\prime}+ v_{x}C_{s} \right)
+Q_{f}(v_{y}\beta_{y}^{\ast} + v_{z}\beta_{z}^{\ast})
-A_{f}b_{\perp}^{\ast}\beta_{\perp}^{\ast 2}/\rho,
\end{equation}
\begin{equation}
R_{57} = \alpha_{f} \left( H^{\prime} + v_{x}C_{f} \right)
-Q_{s}(v_{y}\beta_{y}^{\ast} + v_{z}\beta_{z}^{\ast})
+A_{s}b_{\perp}^{\ast}\beta_{\perp}^{\ast 2}/\rho.
\end{equation}
where $H^{\prime} = H-b^{2}/\rho$.  In these equations
\begin{equation}
\beta_{y}^{\ast} = b_{y}/|b_{\perp}^{\ast}|, \hspace*{1cm}
\beta_{z}^{\ast} = b_{z}/|b_{\perp}^{\ast}|, \hspace*{1cm}
\beta_{\perp}^{\ast 2} = \beta_{y}^{\ast 2} + \beta_{z}^{\ast 2}.
\end{equation}
The left-eigenvectors are the rows of the matrix
\begin{equation}
{\sf L} = \left[ \begin{array}{ccccccc}
L_{11} & -\bar{V}_{xf}-\hat{C}_{ff} & -\bar{V}_{yf} + \hat{Q}_{s}Q_{y}^{\ast} & -\bar{V}_{zf} + \hat{Q}_{s}Q_{z}^{\ast} & \bar{\alpha}_{f} & \hat{A}_{s}{Q}_{y}^{\ast} - \bar{\alpha}_{f}b_{y} & \hat{A}_{s}{Q}_{z}^{\ast} - \bar{\alpha}_{f}b_{z} \\
L_{21} & 0 & -\beta_{z}/2 & \beta_{y}/2 & 0 & -\beta_{z}S\sqrt{\rho}/2 & \beta_{y}S\sqrt{\rho}/2 \\
L_{31} & -\bar{V}_{xs}-\hat{C}_{ss} & -\bar{V}_{ys} - \hat{Q}_{f}Q_{y}^{\ast} & -\bar{V}_{zs} - \hat{Q}_{f}Q_{z}^{\ast} & \bar{\alpha}_{s} & -\hat{A}_{f}{Q}_{y}^{\ast} - \bar{\alpha}_{s}b_{y} & -\hat{A}_{f}{Q}_{z}^{\ast} - \bar{\alpha}_{s}b_{z} \\
L_{41} & 2\bar{v}_{x} & 2\bar{v}_{y} & 2\bar{v}_{z} & -\gamma^{\prime}/a^{2} & 2\bar{b}_{y} & 2\bar{b}_{z} \\
L_{51} & -\bar{V}_{xs}+\hat{C}_{ss} & -\bar{V}_{ys} + \hat{Q}_{f}Q_{y}^{\ast} & -\bar{V}_{zs} + \hat{Q}_{f}Q_{z}^{\ast} & \bar{\alpha}_{s} & -\hat{A}_{f}{Q}_{y}^{\ast} - \bar{\alpha}_{s}b_{y} & -\hat{A}_{f}{Q}_{z}^{\ast} - \bar{\alpha}_{s}b_{z} \\
L_{61} & 0 & \beta_{z}/2 & -\beta_{y}/2 & 0 & -\beta_{z}S\sqrt{\rho}/2 & \beta_{y}S\sqrt{\rho}/2 \\
L_{71} & -\bar{V}_{xf}+\hat{C}_{ff} & -\bar{V}_{yf} - \hat{Q}_{s}Q_{y}^{\ast} & -\bar{V}_{zf} - \hat{Q}_{s}Q_{z}^{\ast} & \bar{\alpha}_{f} & \hat{A}_{s}{Q}_{y}^{\ast} - \bar{\alpha}_{f}b_{y} & \hat{A}_{s}{Q}_{z}^{\ast} - \bar{\alpha}_{f}b_{z}
\end{array} \right]
\end{equation}
where a symbol over the quantity $q$ denotes normalization via
$\bar{q} = \gamma^{\prime}q/(2a^{2})$ or $\hat{q} = q/(2a^{2})$.  In addition,
\begin{equation}
Q_{y}^{\ast} = \beta_{y}^{\ast}/\beta_{\perp}^{\ast 2}, \hspace*{1cm}
Q_{z}^{\ast} = \beta_{z}^{\ast}/\beta_{\perp}^{\ast 2},
\end{equation}
and
\begin{equation}
L_{11} = \bar{\alpha}_{f}(v^{2} - H^{\prime}) + \hat{C}_{ff}(C_{f}+v_{x})
- \hat{Q}_{s}(v_{y}Q_{y}^{\ast} + v_{z}Q_{z}^{\ast})
- \hat{A}_{s}|b_{\perp}|/\rho,
\end{equation}
\begin{equation}
L_{21} = (v_{y}\beta_{z} - v_{z}\beta_{y})/2 = -L_{61}
\end{equation}
\begin{equation}
L_{31} = \bar{\alpha}_{s}(v^{2} - H^{\prime}) + \hat{C}_{ss}(C_{s}+v_{x})
+ \hat{Q}_{f}(v_{y}Q_{y}^{\ast} + v_{z}Q_{z}^{\ast})
+ \hat{A}_{f}|b_{\perp}|/\rho,
\end{equation}
\begin{equation}
L_{41} = 1 - \bar{v}^{2} + 2\hat{X}^{\prime}
\end{equation}
\begin{equation}
L_{51} = \bar{\alpha}_{s}(v^{2} - H^{\prime}) + \hat{C}_{ss}(C_{s}-v_{x})
- \hat{Q}_{f}(v_{y}Q_{y}^{\ast} + v_{z}Q_{z}^{\ast})
+ \hat{A}_{f}|b_{\perp}|/\rho,
\end{equation}
\begin{equation}
L_{71} = \bar{\alpha}_{f}(v^{2} - H^{\prime}) + \hat{C}_{ff}(C_{f}-v_{x})
+ \hat{Q}_{s}(v_{y}Q_{y}^{\ast} + v_{z}Q_{z}^{\ast})
- \hat{A}_{s}|b_{\perp}|/\rho,
\end{equation}

\subsection{Isothermal Magnetohydrodynamics}

For isothermal MHD, ${\bf U} = (\rho, \rho v_{x}, \rho v_{y},
\rho v_{z}, b_{y}, b_{z})$, where ${\bf b} = {\bf B}/\sqrt{4\pi}$,
and the matrix ${\sf A}$ is
\begin{equation}
{\sf A} = \left[ \begin{array}{cccccc}
0 & 1 & 0 & 0 & 0 & 0 \\
- v_{x}^2 + C^2 + X & 2 v_{x} & 0 & 0 & b_{y}Y & b_{z}Y  \\
- v_{x} v_{y} & v_{y} & v_{x} & 0 & -b_{x} & 0 \\
- v_{x} v_{z} & v_{z} & 0 & v_{x} & 0 & -b_{x} \\
(b_{x}v_{y} - b_{y}v_{x})/\rho & b_{y}/\rho & -b_{x}/\rho & 0 & v_{x} & 0 \\
(b_{x}v_{z} - b_{z}v_{x})/\rho & b_{z}/\rho & 0 & -b_{x}/\rho & 0 & v_{x}  
\end{array} \right]
\end{equation}
where $C$ is the isothermal sound speed, and $X$ and $Y$ are given by equations
(B15) and (B16).  The six eigenvalues of this matrix in ascending order are
\begin{equation}
{\bf \lambda} = (v_{x}-C_{f}, v_{x}-C_{Ax}, v_{x}-C_{s},
v_{x}+C_{s}, v_{x}+C_{Ax}, v_{x}+C_{f})
\end{equation}
where the fast- and slow-magnetosonic wave speeds are given by
\begin{equation}
C_{f,s}^2 = \frac{1}{2} \left( \left[\tilde{C}^{2}+\tilde{C}_{A}^{2}\right] \pm
\sqrt{\left[\tilde{C}^{2}+\tilde{C}_{A}^{2}\right]^{2} - 
4\tilde{C}^2 C_{Ax}^2}\right)
\end{equation}
(with $C_{f}[C_{s}]$ given by the $+[-]$ sign), where $\tilde{C}^{2} = C^{2}+X$,
and the Alfv\'{e}n speeds are 
\begin{equation}
\tilde{C}_{A}^{2} = C_{Ax}^{2} + b_{\perp}^{\ast 2}/\rho, \hspace*{1cm}
C_{Ax}^{2} = b_{x}^{2}/\rho, \hspace*{1cm}
b_{\perp}^{\ast 2} = Y(b_{y}^{2} + b_{z}^{2}).
\end{equation}
The corresponding right-eigenvectors are the columns of the matrix given by
equation (B21) with the fifth row and fourth column dropped, and $a$ replaced
by $\tilde{C}$ in the definitions given in equations (A11)-(A12).
The left-eigenvectors are the rows of the matrix
\begin{equation}
{\sf L} = \left[ \begin{array}{cccccc}
L_{11} & -\hat{C}_{ff} & \hat{Q}_{s}Q_{y}^{\ast} & \hat{Q}_{s}Q_{z}^{\ast} & \hat{A}_{s}Q_{y}^{\ast} & \hat{A}_{s}Q_{z}^{\ast} \\
(v_{y}\beta_{z} - v_{z}\beta_{y})/2 & 0 & -\beta_{z}/2 & \beta_{y}/2 & -\beta_{z}S\sqrt{\rho}/2 & \beta_{y}S\sqrt{\rho}/2 \\
L_{31} & -\bar{C}_{ss} & -\bar{Q}_{f}Q_{y}^{\ast} & -\bar{Q}_{f}Q_{z}^{\ast} & -\bar{A}_{f}Q_{y}^{\ast} & -\bar{A}_{f}Q_{z}^{\ast} \\
L_{41} & \bar{C}_{ss} & \bar{Q}_{f}Q_{y}^{\ast} & \bar{Q}_{f}Q_{z}^{\ast} & -\bar{A}_{f}Q_{y}^{\ast} & -\bar{A}_{f}Q_{z}^{\ast} \\
-(v_{y}\beta_{z} - v_{z}\beta_{y})/2 & 0 & \beta_{z}/2 & -\beta_{y}/2 & -\beta_{z}S\sqrt{\rho}/2 & \beta_{y}S\sqrt{\rho}/2 \\
L_{61} & \hat{C}_{ff} & -\hat{Q}_{s}Q_{y}^{\ast} & -\hat{Q}_{s}Q_{z}^{\ast} & \hat{A}_{s}Q_{y}^{\ast} & \hat{A}_{s}Q_{z}^{\ast} \\
\end{array} \right]
\end{equation}
where $C_{ff,ss}, Q_{f,s}$, and $A_{f,s}$ are given by equations (A11)-(A13)
(with $a$ replaced by $C$), $\beta_{y,z}$ are given by equation (B10),
$Q_{y,z}^{\ast}$ are given by equation (B20), and
\begin{equation}
L_{11} = \hat{C}_{ff}(C_{f} + v_{x}) 
- \hat{Q}_{s}(v_{y}Q_{y}^{\ast} + v_{z}Q_{z}^{\ast})
- \hat{A}_{s}|b^{\ast}_{\perp}|/\rho,
\end{equation}
\begin{equation}
L_{31} = \bar{C}_{ss}(C_{s} + v_{x}) 
+ \bar{Q}_{f}(v_{y}Q_{y}^{\ast} + v_{z}Q_{z}^{\ast})
+ \bar{A}_{f}|b^{\ast}_{\perp}|/\rho,
\end{equation}
\begin{equation}
L_{41} = \bar{C}_{ss}(C_{s} - v_{x}) 
- \bar{Q}_{f}(v_{y}Q_{y}^{\ast} + v_{z}Q_{z}^{\ast})
+ \bar{A}_{f}|b^{\ast}_{\perp}|/\rho,
\end{equation}
\begin{equation}
L_{61} = \hat{C}_{ff}(C_{f} - v_{x}) 
+ \hat{Q}_{s}(v_{y}Q_{y}^{\ast} + v_{z}Q_{z}^{\ast})
- \hat{A}_{s}|b^{\ast}_{\perp}|/\rho.
\end{equation}
In these equations, a symbol over the quantity $q$ denotes normalization via
$\bar{q} = q/(C^{2}[1+\alpha_{f}^{2}])$ and
$\hat{q} = q/(C^{2}[1+\alpha_{s}^{2}])$.



\end{document}

